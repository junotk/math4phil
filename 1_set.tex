\documentclass[11pt,a4paper]{jsarticle}
%
\usepackage{amsmath,amssymb}
\usepackage{bm}
\usepackage[dvipdfmx]{graphicx}	% required for `\includegraphics' (yatex added)
\usepackage{setspace}           % required for `\doublespace'
\usepackage{natbib}
\usepackage{ascmac}
\usepackage[margin=30truemm, left=40truemm, right=40truemm]{geometry}
%
% \setlength{\textwidth}{\fullwidth}
% \setlength{\textheight}{39\baselineskip}
% \addtolength{\textheight}{\topskip}
% \setlength{\voffset}{-0.5in}
% \setlength{\topmargin}{0pt}
% \setlength{\headheight}{0pt}
% \setlength{\headsep}{0.3in}

\setlength{\parskip}{3mm} %段落間にスペースを入れる


%%\requirepackage{amssymb}
%\requirepackage{amsmath}
\usepackage{color}


\newtheorem{exercise}{練習問題}[section]
\newtheorem{example}{事例}[section]
\newtheorem{attn}{注意}[section]
\renewcommand{\theattn}{}


% Bold face capital letters:
\newcommand{\bfzero}{\boldsymbol{0}}
\newcommand{\bfone}{\boldsymbol{1}}
\newcommand{\bfA}{\boldsymbol{A}}
\newcommand{\bfB}{\boldsymbol{B}}
\newcommand{\bfC}{\boldsymbol{C}}
\newcommand{\bfD}{\boldsymbol{D}}
\newcommand{\bfE}{\boldsymbol{E}}
\newcommand{\bfF}{\boldsymbol{F}}
\newcommand{\bfG}{\boldsymbol{G}}
\newcommand{\bfH}{\boldsymbol{H}}
\newcommand{\bfI}{\boldsymbol{I}}
\newcommand{\bfJ}{\boldsymbol{J}}
\newcommand{\bfK}{\boldsymbol{K}}
\newcommand{\bfL}{\boldsymbol{L}}
\newcommand{\bfM}{\boldsymbol{M}}
\newcommand{\bfN}{\boldsymbol{N}}
\newcommand{\bfO}{\boldsymbol{O}}
\newcommand{\bfP}{\boldsymbol{P}}
\newcommand{\bfQ}{\boldsymbol{Q}}
\newcommand{\bfR}{\boldsymbol{R}}
\newcommand{\bfS}{\boldsymbol{S}}
\newcommand{\bfT}{\boldsymbol{T}}
\newcommand{\bfU}{\boldsymbol{U}}
\newcommand{\bfV}{\boldsymbol{V}}
\newcommand{\bfW}{\boldsymbol{W}}
\newcommand{\bfX}{\boldsymbol{X}}
\newcommand{\bfY}{\boldsymbol{Y}}
\newcommand{\bfZ}{\boldsymbol{Z}}

\newcommand{\bfa}{\boldsymbol{a}}
\newcommand{\bfb}{\boldsymbol{b}}
\newcommand{\bfc}{\boldsymbol{c}}
\newcommand{\bfd}{\boldsymbol{d}}
\newcommand{\bfe}{\boldsymbol{e}}
\newcommand{\bff}{\boldsymbol{f}}
\newcommand{\bfk}{\boldsymbol{k}}
\newcommand{\bfm}{\boldsymbol{m}}
\newcommand{\bfn}{\boldsymbol{n}}
\newcommand{\bfo}{\boldsymbol{o}}
\newcommand{\bfp}{\boldsymbol{p}}
\newcommand{\bfq}{\boldsymbol{q}}
\newcommand{\bfr}{\boldsymbol{r}}
\newcommand{\bfs}{\boldsymbol{s}}
\newcommand{\bft}{\boldsymbol{t}}
\newcommand{\bfu}{\boldsymbol{u}}
\newcommand{\bfv}{\boldsymbol{v}}
\newcommand{\bfw}{\boldsymbol{w}}
\newcommand{\bfx}{\boldsymbol{x}}
\newcommand{\bfy}{\boldsymbol{y}}
\newcommand{\bfz}{\boldsymbol{z}}



% BB (???) capital letters:
\newcommand{\bbA}{\mathbb{A}}
\newcommand{\bbB}{\mathbb{B}}
\newcommand{\bbC}{\mathbb{C}}
\newcommand{\bbD}{\mathbb{D}}
\newcommand{\bbE}{\mathbb{E}}
\newcommand{\bbF}{\mathbb{F}}
\newcommand{\bbG}{\mathbb{G}}
\newcommand{\bbI}{\mathbb{I}}
\newcommand{\bbN}{\mathbb{N}}
\newcommand{\bbP}{\mathbb{P}}
\newcommand{\bbQ}{\mathbb{Q}}
\newcommand{\bbR}{\mathbb{R}}
\newcommand{\bbU}{\mathbb{U}}
\newcommand{\bbV}{\mathbb{V}}
\newcommand{\bbX}{\mathbb{X}}
\newcommand{\bbY}{\mathbb{Y}}
\newcommand{\bbZ}{\mathbb{Z}}
\newcommand{\bbone}{{\ifmmode\mathrm{1\!l}\else\mbox{\(\mathrm{1\!l}\)}\fi}}


% Caligraphic math capital letters:
\newcommand{\mcalA}{\mathcal{A}}
\newcommand{\mcalB}{\mathcal{B}}
\newcommand{\mcalC}{\mathcal{C}}
\newcommand{\mcalD}{\mathcal{D}}
\newcommand{\mcalE}{\mathcal{E}}
\newcommand{\mcalF}{\mathcal{F}}
\newcommand{\mcalG}{\mathcal{G}}
\newcommand{\mcalH}{\mathcal{H}}
\newcommand{\mcalI}{\mathcal{I}}
\newcommand{\mcalJ}{\mathcal{J}}
\newcommand{\mcalK}{\mathcal{K}}
\newcommand{\mcalL}{\mathcal{L}}
\newcommand{\mcalM}{\mathcal{M}}
\newcommand{\mcalN}{\mathcal{N}}
\newcommand{\mcalO}{\mathcal{O}}
\newcommand{\mcalP}{\mathcal{P}}
\newcommand{\mcalQ}{\mathcal{Q}}
\newcommand{\mcalS}{\mathcal{S}}
\newcommand{\mcalT}{\mathcal{T}}
\newcommand{\mcalU}{\mathcal{U}}
\newcommand{\mcalV}{\mathcal{V}}
\newcommand{\mcalX}{\mathcal{X}}
\newcommand{\mcalY}{\mathcal{Y}}
\newcommand{\mcalZ}{\mathcal{Z}}

% Graph nodes notations:
\newcommand{\PA}{\mathit{PA}}
\newcommand{\bfPA}{\mathbf{PA}}
\newcommand{\CH}{\mathit{CH}}
\newcommand{\bfCH}{\mathbf{CH}}
\newcommand{\DS}{\mathit{DS}}
\newcommand{\bfDS}{\mathbf{DS}}
\newcommand{\ND}{\mathit{ND}}
\newcommand{\bfND}{\mathbf{ND}}
\newcommand{\AN}{\mathit{an}}
\newcommand{\bfAN}{\mathbf{an}}
\newcommand{\pa}{\mathit{pa}}
\newcommand{\bfpa}{\mathbf{pa}}
\newcommand{\ch}{\mathit{ch}}
\newcommand{\bfch}{\mathbf{ch}}
\newcommand{\ds}{\mathit{ds}}
\newcommand{\bfds}{\mathbf{ds}}
\newcommand{\nd}{\mathit{nd}}
\newcommand{\bfnd}{\mathbf{nd}}
\newcommand{\an}{\mathit{an}}
\newcommand{\bfan}{\mathbf{an}}



\DeclareMathOperator*{\argmax}{arg\,max}
\DeclareMathOperator*{\argmin}{arg\,min}
\DeclareMathOperator*{\argsup}{arg\,sup}
\DeclareMathOperator*{\arginf}{arg\,inf}
\DeclareMathOperator{\erfc}{erfc}
\DeclareMathOperator{\diag}{diag}
\DeclareMathOperator{\cum}{cum}
\DeclareMathOperator{\sgn}{sgn}
\DeclareMathOperator{\tr}{tr}
\DeclareMathOperator{\spn}{span}
\DeclareMathOperator{\adj}{adj}
\DeclareMathOperator{\E}{\mathbb{E}}
\DeclareMathOperator{\var}{Var}
\DeclareMathOperator{\cov}{Cov}
\DeclareMathOperator{\corr}{corr}
\DeclareMathOperator{\sech}{sech}
\DeclareMathOperator{\sinc}{sinc}
\DeclareMathOperator*{\lms}{l.i.m.\,}
\newcommand{\varop}[1]{\var\left[{#1}\right]}
\newcommand{\covop}[2]{\cov\left({#1},{#2}\right)}
\newcommand{\T}{^\textrm{T}}
\newcommand\indep{\protect\mathpalette{\protect\independenT}{\perp}}
\def\independenT#1#2{\mathrel{\rlap{$#1#2$}\mkern2mu{#1#2}}}

\newcommand{\bfalpha}{\boldsymbol{\alpha}}
\newcommand{\bfbeta} {\boldsymbol{\beta}}
\newcommand{\bfgamma}{\boldsymbol{\gamma}}
\newcommand{\bfeta}  {\boldsymbol{\eta}}
\newcommand{\bftheta}{\boldsymbol{\theta}}
\newcommand{\bflambda}   {\boldsymbol{\lambda}}
\newcommand{\bfmu}   {\boldsymbol{\mu}}
\newcommand{\bfnu}   {\boldsymbol{\nu}}
\newcommand{\bfxi}   {\boldsymbol{\xi}}
\newcommand{\bfpsi}  {\boldsymbol{\psi}}
\newcommand{\bfphi}   {\boldsymbol{\phi}}
\newcommand{\bfrho}   {\boldsymbol{\rho}}
\newcommand{\bfvarepsilon}{\boldsymbol{\varepsilon}}
%\newcommand{\qed}{{qed}}
%\newcommand{\eqalignno}[1]{\begin{array}{ccccccc}#1\end{array}}

\newcommand{\bfGamma}{\boldsymbol{\Gamma}}
\newcommand{\bfTheta}{\boldsymbol{\Theta}}
\newcommand{\bfLambda}   {\boldsymbol{\Lambda}}
\newcommand{\bfPsi}  {\boldsymbol{\Psi}}
\newcommand{\bfPhi}   {\boldsymbol{\Phi}}
\newcommand{\bfSigma}  {\boldsymbol{\Sigma}}
\newcommand{\bfOmega}  {\boldsymbol{\Omega}}


% DISTRIBUTIOoNS: 
\newcommand{\normal}{\mathcal{N}}
\newcommand{\binomial}{\mathcal{B}}
\newcommand{\multinomial}{\mathcal{M}}
\newcommand{\exponential}{\mathcal{E}}
\newcommand{\geometric}{\mathcal{G}}
\newcommand{\poisson}{\mbox{Poisson}}
\newcommand{\uniform}{\mbox{Uniform}}



% Theorems
\newtheorem{postulate}{Postulate}
\newtheorem{theorem}{Theorem}
\newtheorem{corollary}{Corollary}
\newtheorem{lemma}{Lemma}
\newtheorem{definition}{Definition}

%PSTricks (commande for latent nodes)
\newcommand{\lnode}[4]{ \cnode(#1){#2}{#3}\rput(#1){\footnotesize#4} }

% KEEPING TRACK OF WORK
\newcommand{\todo}[1]
{
{\color{red}{
[TODO: #1]}}
\addcontentsline{toc}{subsection}{TO DO: #1}
}

\newcommand{\fixme}[1]{{\color{red}{#1}}}

\newcommand{\new}[1]{{\color{blue}{#1}}}


\newcommand{\note}[2]
{
{\color{red}{
[#1: #2]}}
}




\makeatletter
% define \citepos for posesive citation (e.g. Otsuka's (2015))
\DeclareRobustCommand\citepos
  {\begingroup
   \let\NAT@nmfmt\NAT@posfmt% ...except with a different name format
   \NAT@swafalse\let\NAT@ctype\z@\NAT@partrue
   \@ifstar{\NAT@fulltrue\NAT@citetp}{\NAT@fullfalse\NAT@citetp}}

\let\NAT@orig@nmfmt\NAT@nmfmt
\def\NAT@posfmt#1{\NAT@orig@nmfmt{#1's}}
\makeatother


% \pagestyle{myheadings}
% \markright{\footnotesize \sf 2022秋期「哲学者のための数学」授業資料(大塚淳) \ \ 配布禁止}
\begin{document}


\title{1. 集合}
\author{2022秋期「哲学者のための数学」授業資料(大塚淳)}
\date{ver. \today}
\maketitle

\section{集合の基礎}
% 集合の定義
集合(set)とは,その名の通りモノの集まりである.
集合の中に入っているモノをその元ないし要素(element)と呼ぶ.
つまり集合は元の集まりである.

\subsection{集合の定義}

集合を定義するには二つのやり方がある.\emph{外延的}(extensive)な定義では,元を並べることによって集合を定義する:
\[
 A := \{\text{John, Paul, George, Ringo}\}.
\]
ちなみに「$:=$」は左辺を右辺で定義する,という意味で使われる.
ここで$A$は4つの元を持つ集合として定義されている.
同じことを\emph{内包的}(intensive)に表すこともできる
\[
 A := \{x | x \text{ is a member of the Beatles}\}.
\]
「a member of the Beatles」であるような$x$をすべて集めてきた集合,というように理解すると良い.
内包的定義はいちいち元を明示しなくて良いので便利である.例えば
\[
 \mathbb{N} := \{x | x \text{ is a natural number}\}
\]
は無限個の元を持つ集合だが,これを外延的に記述するのは不可能だろう.
ただし,内包的定義に用いる述語は曖昧でなく,それが成り立つかどうかが明確に決まるものでなければならない.例えば,
\[
 T := \{x | x \text{ は背が高い}\}
\]
は,何センチ以上なら「背が高い」とされるかが明示的に決められていない限り,集合の定義として機能しない.
逆に言うと,適用基準が明確な述語であれば対応する集合を考えることができる(ただ後で見るように,内包的定義を無制約に用いると重大なパラドクスを生んでしまうこともある).
また以下のように方程式の解を集合で表すこともある:
\[
 \{ x | x^2 = 4\} = \{ -2, 2 \}.
\]


集合は元として集合をとることもできる.なので以下も集合である
\[
\{ \{\text{John, Paul, George, Ringo}\}, \{ \text{Jackie, Tito, Jermaine, Marlon, Michael}\}, \text{James} \}.
\]
これは二つのグループ(人の集合)と一人の歌手からなる集合である.
元と集合をしっかりと区別すること.例えば
\[
\{\text{John, Paul, George, Ringo}\} \neq \{ \{\text{John, Paul, George, Ringo}\} \},
\]
特に
\[
\{\text{James}\} \neq \{ \{\text{James}\} \}  
\]
であることに注意(なぜそうなのか,説明してみよう).


\subsection{要素関係と部分関係}
集合において我々が問うべきことは唯一つ,それがしかじかのものを元として含むか否か,ということだけである.
実際,集合論におけるすべての問題は,最終的には,その集合に何が入っているかという話に行き着く.
ある対象$a$が集合$A$の要素であることを以下のように書く:
\[
 a \in A.
\]
なので例えば
\begin{align*}
\text{Michael} &\in \{ \text{Jackie, Tito, Jermaine, Marlon, Michael}\}  \\
3 &\in \mathbb{N} \\
\{ \text{Michael} \} &\not\in \{ \text{Jackie, Tito, Jermaine, Marlon, Michael}\}  \\
\end{align*}
である.
特に最初と最後の違いに気をつけよう.
Michael JacksonはJackson 5のメンバーであるが,Michael個人からなる集合はそうではない.

要素関係$\in$は要素と集合の間に成り立つ関係だった.
一方,集合と集合の間の関係を考えることができる.
2つの集合$A, B$があったとき,$A$が$B$の\emph{部分集合}(subset)であるとは,$A$の元がすべて$B$に含まれるとき,すなわち
\[
 \forall x (x \in A \Rightarrow x \in B)
\]
が成立するときである.これを$A \subset B$と書く.
図的には,これは$A$の集合の範囲がすっぽり$B$の範囲に含まれている事態に相当する.

2つの集合$A,B$が\emph{等しい}とは,両者が互いの部分集合であるとき,つまり
\begin{align*}
 A = B &\iff A \subset B \wedge B \subset A \\
&\iff \forall x (x \in A \iff x \in B)
\end{align*}
 が成立することである.
このとき,$A$の範囲はぴったり$B$と一致する.
上の同一性の定義から,{元の個数や並び方は集合の同一性に影響しない}ということが帰結する.
なので例えば
\begin{align*}
 \{\text{John, Paul, George, Ringo}\} 
 &= \{\text{Ringo, Paul, John, George}\} \\
 &= \{\text{Ringo, Ringo, Paul, John, George, John}\}
\end{align*}
である(上の同一性の定義から確認せよ).
つまり,集合の「本質」を決めるのはそこに何が入っているかであって,何個/どの順で入っているか,ということは全く考慮されない.

$A \subset B$であるが$A \neq B$であるとき,$A$は$B$の\emph{真部分集合}(proper/strict subset)であるといい,これを$A \subsetneq B$と書くこともある(が本講義ではあまり使わない).
きちんと書き下せばこれはつまり
\[
A \subsetneq B \iff \forall x (x \in A \Rightarrow x \in B) \wedge \exists x (x \not\in A \wedge x \in B)
\]
ということ,つまり$A$の元はすべて$B$に含まれるが,$A$になくて$B$にはある元が少なくとも1つはある,ということである.

% empty set
いかなる元も持たない集合を\emph{空集合}(empty set)といい,これを$\{\}$または$\emptyset$で表す.
いかなるものも空集合の元ではないので,
\[
\forall x (x \not\in \emptyset)
\]
が成り立つ.また空集合は任意の集合の部分集合である,なぜなら任意の集合$A$について.
\[
\forall x (x \in \emptyset \Rightarrow x \in A)
\]
は,どんな$x$に対してもカッコ内の前件が偽($x \not\in \emptyset$)であるがゆえに常に成立するからだ.

% 空集合を内包的に定義してみよ.

% 注意
要素関係$\in$と部分集合関係$\subset$を混同しないように!
前者は要素と集合の間の関係,後者は集合と集合の間の関係である.
だから例えば
\[
\text{Michael} \subset \{ \text{Jackie, Tito, Jermaine, Marlon, Michael}\}  
\]
という式は文法的に正しい形ではない.
では次の式はどうだろうか,
\begin{align*}
\{ \text{Michael} \} &\subset \{ \text{Jackie, Tito, Jermaine, Marlon, Michael}\}  \\
\{ \text{Michael} \} &\in \{ \text{Jackie, Tito, Jermaine, Marlon, Michael}\}  \\
\{  \} &\in \{ \text{Jackie, Tito, Jermaine, Marlon, Michael}\}  
\end{align*}
それぞれ真/偽/文法的に不正かどうか考えてみよう.






%
% 種は集合として定義できるだろうか


\bibliographystyle{apalike}
%\bibliography{/Users/jun/Dropbox/settings/library.bib}


\end{document}