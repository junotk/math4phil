\documentclass[11pt,a4paper]{jsarticle}
\usepackage{amsmath,amssymb}
\usepackage{amsthm}
\usepackage{ascmac}
\usepackage{bm}
\usepackage[dvipdfmx]{graphicx}	% required for `\includegraphics' (yatex added)
\usepackage{setspace}           % required for `\doublespace'
\usepackage{tikz}
\usepackage{tikz-cd}
\usetikzlibrary{angles, positioning, shapes, arrows.meta, decorations.pathmorphing}
%\usetikzlibrary{intersections, calc, arrows, positioning, arrows.meta}
\usepackage{tcolorbox}  % 定理環境の装飾
\tcbuselibrary{skins, breakable, theorems}
\usepackage{xcolor}
\usepackage{natbib}
\usepackage{pxrubrica}
\usepackage[margin=30truemm, left=40truemm, right=40truemm]{geometry}
\usepackage{thmbox}     % required for theorem environment with side bar
%
\setlength{\parskip}{3mm} %段落間にスペースを入れる


% \pagestyle{myheadings}
% \markright{\footnotesize \sf 2022秋期「哲学者のための数学」授業資料(大塚淳) \ \ 配布禁止}


\theoremstyle{definition}
\newtheorem[S]{exercise}{練習問題}[section]
\newtheorem[S]{example}{事例}[section]
\newtheorem[S]{fact}{事実}[section]
\newtheorem[S]{attn}{注意}[section]
\newtheorem[S]{develop}{発展}[section]
\renewcommand{\theattn}{}

\newtcbtheorem[auto counter, number within=section]{rei}{事例}{
    breakable,
    coltitle=black,
    fonttitle=\bfseries,
    enhanced, colback=white, frame hidden, borderline west = {0.5pt}{5pt}{black},
%    number freestyle={\noexpand\thesection.\noexpand\arabic{\tcbcounter}}
}{rei}

\newtcbtheorem[auto counter, number within=section]{prop}{命題}{
    breakable,
    coltitle=black,
    fonttitle=\bfseries,
    enhanced, colback=white, frame hidden, borderline west = {0.5pt}{5pt}{black},
%    number freestyle={\noexpand\thesection.\noexpand\arabic{\tcbcounter}}
}{prop}

\newtcbtheorem[number within=section]{renshu}{練習問題}{
    breakable,
    coltitle=black,
    fonttitle=\bfseries,
    enhanced, colback=white, frame hidden, borderline west = {0.5pt}{5pt}{black}
}{renshu}


\newtcbtheorem[number within=section]{hatten}{発展}{
    breakable,
    coltitle=black,
    fonttitle=\bfseries,
    enhanced, colback=white, frame hidden, borderline west = {0.5pt}{5pt}{black}
}{renshu}


\newtcbtheorem[number within=section]{dfn}{定義}{
    fonttitle=\bfseries,
    enhanced, colback=white
}{dfn}


% Bold face capital letters:
\newcommand{\bfzero}{\boldsymbol{0}}
\newcommand{\bfone}{\boldsymbol{1}}
\newcommand{\bfA}{\boldsymbol{A}}
\newcommand{\bfB}{\boldsymbol{B}}
\newcommand{\bfC}{\boldsymbol{C}}
\newcommand{\bfD}{\boldsymbol{D}}
\newcommand{\bfE}{\boldsymbol{E}}
\newcommand{\bfF}{\boldsymbol{F}}
\newcommand{\bfG}{\boldsymbol{G}}
\newcommand{\bfH}{\boldsymbol{H}}
\newcommand{\bfI}{\boldsymbol{I}}
\newcommand{\bfJ}{\boldsymbol{J}}
\newcommand{\bfK}{\boldsymbol{K}}
\newcommand{\bfL}{\boldsymbol{L}}
\newcommand{\bfM}{\boldsymbol{M}}
\newcommand{\bfN}{\boldsymbol{N}}
\newcommand{\bfO}{\boldsymbol{O}}
\newcommand{\bfP}{\boldsymbol{P}}
\newcommand{\bfQ}{\boldsymbol{Q}}
\newcommand{\bfR}{\boldsymbol{R}}
\newcommand{\bfS}{\boldsymbol{S}}
\newcommand{\bfT}{\boldsymbol{T}}
\newcommand{\bfU}{\boldsymbol{U}}
\newcommand{\bfV}{\boldsymbol{V}}
\newcommand{\bfW}{\boldsymbol{W}}
\newcommand{\bfX}{\boldsymbol{X}}
\newcommand{\bfY}{\boldsymbol{Y}}
\newcommand{\bfZ}{\boldsymbol{Z}}

\newcommand{\bfa}{\boldsymbol{a}}
\newcommand{\bfb}{\boldsymbol{b}}
\newcommand{\bfc}{\boldsymbol{c}}
\newcommand{\bfd}{\boldsymbol{d}}
\newcommand{\bfe}{\boldsymbol{e}}
\newcommand{\bff}{\boldsymbol{f}}
\newcommand{\bfk}{\boldsymbol{k}}
\newcommand{\bfm}{\boldsymbol{m}}
\newcommand{\bfn}{\boldsymbol{n}}
\newcommand{\bfo}{\boldsymbol{o}}
\newcommand{\bfp}{\boldsymbol{p}}
\newcommand{\bfq}{\boldsymbol{q}}
\newcommand{\bfr}{\boldsymbol{r}}
\newcommand{\bfs}{\boldsymbol{s}}
\newcommand{\bft}{\boldsymbol{t}}
\newcommand{\bfu}{\boldsymbol{u}}
\newcommand{\bfv}{\boldsymbol{v}}
\newcommand{\bfw}{\boldsymbol{w}}
\newcommand{\bfx}{\boldsymbol{x}}
\newcommand{\bfy}{\boldsymbol{y}}
\newcommand{\bfz}{\boldsymbol{z}}



% BB (???) capital letters:
\newcommand{\bbA}{\mathbb{A}}
\newcommand{\bbB}{\mathbb{B}}
\newcommand{\bbC}{\mathbb{C}}
\newcommand{\bbD}{\mathbb{D}}
\newcommand{\bbE}{\mathbb{E}}
\newcommand{\bbF}{\mathbb{F}}
\newcommand{\bbG}{\mathbb{G}}
\newcommand{\bbI}{\mathbb{I}}
\newcommand{\bbN}{\mathbb{N}}
\newcommand{\bbP}{\mathbb{P}}
\newcommand{\bbQ}{\mathbb{Q}}
\newcommand{\bbR}{\mathbb{R}}
\newcommand{\bbU}{\mathbb{U}}
\newcommand{\bbV}{\mathbb{V}}
\newcommand{\bbX}{\mathbb{X}}
\newcommand{\bbY}{\mathbb{Y}}
\newcommand{\bbZ}{\mathbb{Z}}
\newcommand{\bbone}{{\ifmmode\mathrm{1\!l}\else\mbox{\(\mathrm{1\!l}\)}\fi}}


% Caligraphic math capital letters:
\newcommand{\mcalA}{\mathcal{A}}
\newcommand{\mcalB}{\mathcal{B}}
\newcommand{\mcalC}{\mathcal{C}}
\newcommand{\mcalD}{\mathcal{D}}
\newcommand{\mcalE}{\mathcal{E}}
\newcommand{\mcalF}{\mathcal{F}}
\newcommand{\mcalG}{\mathcal{G}}
\newcommand{\mcalH}{\mathcal{H}}
\newcommand{\mcalI}{\mathcal{I}}
\newcommand{\mcalJ}{\mathcal{J}}
\newcommand{\mcalK}{\mathcal{K}}
\newcommand{\mcalL}{\mathcal{L}}
\newcommand{\mcalM}{\mathcal{M}}
\newcommand{\mcalN}{\mathcal{N}}
\newcommand{\mcalO}{\mathcal{O}}
\newcommand{\mcalP}{\mathcal{P}}
\newcommand{\mcalQ}{\mathcal{Q}}
\newcommand{\mcalS}{\mathcal{S}}
\newcommand{\mcalT}{\mathcal{T}}
\newcommand{\mcalU}{\mathcal{U}}
\newcommand{\mcalV}{\mathcal{V}}
\newcommand{\mcalX}{\mathcal{X}}
\newcommand{\mcalY}{\mathcal{Y}}
\newcommand{\mcalZ}{\mathcal{Z}}

% Graph nodes notations:
\newcommand{\PA}{\mathit{PA}}
\newcommand{\bfPA}{\mathbf{PA}}
\newcommand{\CH}{\mathit{CH}}
\newcommand{\bfCH}{\mathbf{CH}}
\newcommand{\DS}{\mathit{DS}}
\newcommand{\bfDS}{\mathbf{DS}}
\newcommand{\ND}{\mathit{ND}}
\newcommand{\bfND}{\mathbf{ND}}
\newcommand{\AN}{\mathit{an}}
\newcommand{\bfAN}{\mathbf{an}}
\newcommand{\pa}{\mathit{pa}}
\newcommand{\bfpa}{\mathbf{pa}}
\newcommand{\ch}{\mathit{ch}}
\newcommand{\bfch}{\mathbf{ch}}
\newcommand{\ds}{\mathit{ds}}
\newcommand{\bfds}{\mathbf{ds}}
\newcommand{\nd}{\mathit{nd}}
\newcommand{\bfnd}{\mathbf{nd}}
\newcommand{\an}{\mathit{an}}
\newcommand{\bfan}{\mathbf{an}}



\DeclareMathOperator*{\argmax}{arg\,max}
\DeclareMathOperator*{\argmin}{arg\,min}
\DeclareMathOperator*{\argsup}{arg\,sup}
\DeclareMathOperator*{\arginf}{arg\,inf}
\DeclareMathOperator{\erfc}{erfc}
\DeclareMathOperator{\diag}{diag}
\DeclareMathOperator{\cum}{cum}
\DeclareMathOperator{\sgn}{sgn}
\DeclareMathOperator{\tr}{tr}
\DeclareMathOperator{\spn}{span}
\DeclareMathOperator{\adj}{adj}
\DeclareMathOperator{\E}{\mathbb{E}}
\DeclareMathOperator{\var}{Var}
\DeclareMathOperator{\cov}{Cov}
\DeclareMathOperator{\corr}{corr}
\DeclareMathOperator{\sech}{sech}
\DeclareMathOperator{\sinc}{sinc}
\DeclareMathOperator*{\lms}{l.i.m.\,}
\newcommand{\varop}[1]{\var\left[{#1}\right]}
\newcommand{\covop}[2]{\cov\left({#1},{#2}\right)}
\newcommand{\T}{^\textrm{T}}
\newcommand\indep{\protect\mathpalette{\protect\independenT}{\perp}}
\def\independenT#1#2{\mathrel{\rlap{$#1#2$}\mkern2mu{#1#2}}}

\newcommand{\bfalpha}{\boldsymbol{\alpha}}
\newcommand{\bfbeta} {\boldsymbol{\beta}}
\newcommand{\bfgamma}{\boldsymbol{\gamma}}
\newcommand{\bfeta}  {\boldsymbol{\eta}}
\newcommand{\bftheta}{\boldsymbol{\theta}}
\newcommand{\bflambda}   {\boldsymbol{\lambda}}
\newcommand{\bfmu}   {\boldsymbol{\mu}}
\newcommand{\bfnu}   {\boldsymbol{\nu}}
\newcommand{\bfxi}   {\boldsymbol{\xi}}
\newcommand{\bfpsi}  {\boldsymbol{\psi}}
\newcommand{\bfphi}   {\boldsymbol{\phi}}
\newcommand{\bfrho}   {\boldsymbol{\rho}}
\newcommand{\bfvarepsilon}{\boldsymbol{\varepsilon}}
%\newcommand{\qed}{{qed}}
%\newcommand{\eqalignno}[1]{\begin{array}{ccccccc}#1\end{array}}

\newcommand{\bfGamma}{\boldsymbol{\Gamma}}
\newcommand{\bfTheta}{\boldsymbol{\Theta}}
\newcommand{\bfLambda}   {\boldsymbol{\Lambda}}
\newcommand{\bfPsi}  {\boldsymbol{\Psi}}
\newcommand{\bfPhi}   {\boldsymbol{\Phi}}
\newcommand{\bfSigma}  {\boldsymbol{\Sigma}}
\newcommand{\bfOmega}  {\boldsymbol{\Omega}}


% DISTRIBUTIOoNS: 
\newcommand{\normal}{\mathcal{N}}
\newcommand{\binomial}{\mathcal{B}}
\newcommand{\multinomial}{\mathcal{M}}
\newcommand{\exponential}{\mathcal{E}}
\newcommand{\geometric}{\mathcal{G}}
\newcommand{\poisson}{\mbox{Poisson}}
\newcommand{\uniform}{\mbox{Uniform}}

% Logic
\newcommand{\true}{\texttt{true}}
\newcommand{\false}{\texttt{false}}


%PSTricks (commande for latent nodes)
\newcommand{\lnode}[4]{ \cnode(#1){#2}{#3}\rput(#1){\footnotesize#4} }

% KEEPING TRACK OF WORK
\newcommand{\todo}[1]
{
{\color{red}{
[TODO: #1]}}
\addcontentsline{toc}{subsection}{TO DO: #1}
}

\newcommand{\fixme}[1]{{\color{red}{#1}}}

\newenvironment{answer}[1]
{\par \color{blue}{#1}}
{}


\newcommand{\note}[2]
{
{\color{red}{
[#1: #2]}}
}




\makeatletter
% define \citepos for posesive citation (e.g. Otsuka's (2015))
\DeclareRobustCommand\citepos
  {\begingroup
   \let\NAT@nmfmt\NAT@posfmt% ...except with a different name format
   \NAT@swafalse\let\NAT@ctype\z@\NAT@partrue
   \@ifstar{\NAT@fulltrue\NAT@citetp}{\NAT@fullfalse\NAT@citetp}}

\let\NAT@orig@nmfmt\NAT@nmfmt
\def\NAT@posfmt#1{\NAT@orig@nmfmt{#1's}}
\makeatother




% Code for drawing color circle used in topology (pathconnectedness)
\usepackage{xparse}
\ExplSyntaxOn

\keys_define:nn { colour_transition_circle } {
    inner   .fp_set:N   = \l__inner_radius,
    inner   .initial:n  = {2},
    outer   .fp_set:N   = \l__outer_radius,
    outer   .initial:n  = {3},
    angle   .fp_set:N   = \l__start_angle,
    angle   .initial:n  = {0}
}

\NewDocumentCommand \ColourTransitionCircle { O{} m } {
\group_begin:
    \keys_set:nn { colour_transition_circle } {#1}
    \clist_clear:N \l_tmpa_clist
    \clist_map_inline:nn {#2} {
        \clist_put_right:Nn \l_tmpa_clist {##1}
        %\clist_put_right:Nn \l_tmpa_clist {##1}
    }
    \exp_args:Nx \col_trans_circ:n \l_tmpa_clist
\group_end:
}

\cs_new_protected:Npn \col_trans_circ:n #1 {
    \int_step_inline:nnnn {1} {1} {\clist_count:n {#1} - 1} {
        \path[top~color=\clist_item:nn {#1} {##1}, bottom~color=\clist_item:nn {#1} {##1+1}, shading~angle={270-(180-360/\clist_count:n {#1})/2+(##1-1)*360/\clist_count:n {#1}+\fp_use:N \l__start_angle}] ({\fp_use:N \l__inner_radius*cos((##1-1)*360/\clist_count:n {#1}+\fp_use:N \l__start_angle)},{\fp_use:N \l__inner_radius*sin((##1-1)*360/\clist_count:n {#1}+\fp_use:N \l__start_angle)}) arc[radius = \fp_use:N \l__inner_radius, start~angle={(##1-1)*360/\clist_count:n {#1}+\fp_use:N \l__start_angle}, delta~angle=360/\clist_count:n {#1}] -- ({\fp_use:N \l__outer_radius*cos(##1*360/\clist_count:n {#1}+\fp_use:N \l__start_angle)},{\fp_use:N \l__outer_radius*sin(##1*360/\clist_count:n {#1}+\fp_use:N \l__start_angle)}) arc[radius = \fp_use:N \l__outer_radius, start~angle={##1*360/\clist_count:n {#1}+\fp_use:N \l__start_angle}, delta~angle=-360/\clist_count:n {#1}] -- cycle;
    }
    \path[top~color=\clist_item:nn {#1} {\clist_count:n {#1}}, bottom~color=\clist_item:nn {#1} {1}, shading~angle={180-180/\clist_count:n {#1}+\fp_use:N \l__start_angle}]({\fp_use:N \l__inner_radius*cos((\clist_count:n {#1}-1)*360/\clist_count:n {#1}+\fp_use:N \l__start_angle)},{\fp_use:N \l__inner_radius*sin((\clist_count:n {#1}-1)*360/\clist_count:n {#1}+\fp_use:N \l__start_angle)}) arc[radius = \fp_use:N \l__inner_radius, start~angle={(\clist_count:n {#1}-1)*360/\clist_count:n {#1}+\fp_use:N \l__start_angle}, delta~angle=360/\clist_count:n {#1}] -- ({\fp_use:N \l__outer_radius*cos(\clist_count:n {#1}*360/\clist_count:n {#1}+\fp_use:N \l__start_angle)},{\fp_use:N \l__outer_radius*sin(\clist_count:n {#1}*360/\clist_count:n {#1}+\fp_use:N \l__start_angle)}) arc[radius = \fp_use:N \l__outer_radius, start~angle={\clist_count:n {#1}*360/\clist_count:n {#1}+\fp_use:N \l__start_angle}, delta~angle=-360/\clist_count:n {#1}] -- cycle;
}

\ExplSyntaxOff


\begin{document}


\title{「哲学者のための数学」練習問題の解答}
\date{ver. \today}
\maketitle


\section{集合}

\subparagraph{3.1}
1 真. 2 偽. 3 真.

\subparagraph{4.1}
1 偽. 2 真. 3 真. 4 真.

\subparagraph{4.2}
$\Omega := \{ x | x=x \}, \emptyset := \{ x | x \neq x \}$.他にも色々ありうる.$\Omega$は条件が恒真式(トートロジー),$\emptyset$は矛盾式になればよい.

\subparagraph{4.3}
$[a] := \{ x | x=a \}$.

\subparagraph{5.1}
2は本文同様なので省略する.3については以下の通り(4は3と同様なので略).
\begin{align*}
x \in (A \cup B)^c &\iff \neg (x \in (A \cup B))  & \because \text{補集合の定義より}\\
&\iff  \neg (x \in A \vee x \in B)                & \because \cup\text{の定義より} \\
&\iff  \neg (x \in A) \wedge  \neg (x \in B)      & \because \text{ド・モルガン則より}\\
&\iff  (x \in A^c) \wedge (x \in B^c)             & \because \text{補集合の定義より}\\
&\iff  x \in (A^c \cap B^c)                       & \because \cap\text{の定義より}
\end{align*}

\subparagraph{5.2}
結合律については本文通り.可換律については,本文同様$A= \{1,2\}, B=\{1\}$とすれば,$A \setminus B = \{2\}$であるのに対し$B \setminus A = \emptyset$となり等しくならないことがわかる.

\subparagraph{8.1}
等しくない.例えば本文同様$A = \{a_1, a_2, a_3\}, B = \{b_1, b_2\}$とすると,$A \times B$の元はすべて$(a_i, b_j)$となるが,$B \times A$の元は$(b_i, a_j)$となり順序が異なる.


\section{関係と関数}


\subparagraph{1.1}
\begin{enumerate}
    \item $\{ (n, m) \in \bbN \times \bbN | \exists a \in \bbN (n = a \cdot m )\}$
    \item $\{ (x, y, z) \in X \times X \times X | y \text{ and } z \text{ are biological parents of } x \}$
\end{enumerate}


\subparagraph{2.1} 
「$=$」は反射的,対称的かつ推移的.「$<$」は推移的.「$\leq$」は反射的かつ推移的.


\subparagraph{2.2}
(1) 名前を知っている (2) 親類である (3) 母である (4) 子孫である.

\subparagraph{3.2}
5つ (北海道,本州,四国,九州,沖縄).

\subparagraph{3.3}
推移的でないため不可能.例:山口/岡山/福岡


\subparagraph{事例3.1}
同値類は一つひとつの可能世界である.

\subparagraph{事例3.2}
推移性を満たさない.
 
\subparagraph{4.1}
\begin{enumerate}
    \item $I_{Np}(t) = \emptyset$ for $t < t_0, t_1 < t$.
    \item 時空的連続性の問題.空間的に離れた部分集合,時間的に連続していない集合が「個物」として認められてしまう.
\end{enumerate}
	 

\subparagraph{5.1}
$f \circ f (x) = f(f(x)) = f(x^3 - 2x) = (x^3 - 2x)^3 - 2(x^3 -2x) = x^9 -6x^7 + 12x^5 - 8x^3$
 

\subparagraph{6.1}
日本:Bob / アメリカ:Alice, Dave / フランス:Chris
 


\subparagraph{6.2}
$[a] = f^{-1}(y) = \{x \in X | f(x) = f(a) \}$.
反射性・対称性・推移性は「$=$」より従う(練習問題2.1).つまり
[反射性]:明らかに$f(a) = f(a)$.
[対称性]:$f(a) = f(b)$ならば$f(b) = f(a)$.
[推移性]:$f(a) = f(b), f(b) = f(c)$ならば$f(a) = f(c)$.

 

\subparagraph{6.3}
この関数に入力する部分集合$A$として単元集合$\{x'\}$をとると,$f(\{x'\}) = \{f(x)|x \in \{x'\} \}$となるが,$\{x'\}$に含まれる要素は$x'$だけなので,これは結局$f(x')$だけからなる単元集合$\{f(x')\}$になる.
よって単元集合の像は単元集合となり,これは元の関数$f:x' \mapsto f(x')$と同一視できる. 


\subparagraph{6.4}
$x' \in \{x'\}$であり,それ以外に$\{x'\}$の要素はないので,$x' \in f^{-1}(f(\{x'\})) = \{ x | f(x) \in f(\{x'\}) \}$を示せばよい.
上の問題より$f(\{x'\}) = \{f (x')\}$なので,$f(x') \in f(\{x'\})$.よって上の条件が満たされ$x' \in f^{-1}(f(\{x'\}))$となる.

%\subparagraph{事例6.3}
 

\subparagraph{7.1}
全射でない,つまりある$y \in Y$に対し$f(x) = y$となる$x$が存在しないとする.すると$f^{-1}(y) = \emptyset$となり,$f^{-1}$が$Y$から$X$への関数にならない(関数は$X$の要素をあてがわねばならない).
 

\subparagraph{7.2}
全射である.任意の$[x] \in X/R$は,そこに含まれる元$x \in [x]$に対して$f(x) = [x]$.
しかし単射ではない.例えば異なる$x \neq y$に対し$xRy$であれば,$x, y \in [x]$となり,よって$f(x) = f(y)$.
単射であるためには,$R$が自分自身のみと成立するとき,つまり$\forall x, y (x \neq y \Rightarrow \neg x R y)$でなければならない.
 


\section{順序}

\subparagraph{2.1}
\begin{enumerate}
    \item 任意の部分集合$A \subset X$に対し,$A \subset A$.また$A, B, C$をそれぞれ$X$の部分集合とし,$A \subset B, B \subset C$を仮定すると,部分集合の定義上,$\forall x(x \in A \Rightarrow x \in C)$がなりたつ.よって$A \subset C$. 
    \item 任意の$n \in \mathbb{B}$について$n=n$であり, また$m, l \in \mathbb{B}$について$n=m$, $m=l$ならば$n=l$なので,前順序である.
    \item 略
    \item どのような人も,「自分自身の祖先である」とは言われないので,反射性が満たされず,前順序ではない.
    \item 「$x$は$y$の部分である(ないし$y$に含まれる)」という関係は前順序である.どのようなもの$x$も,自分自身の部分であるといえる.また$x$が$y$の部分であり,$y$が$z$の部分であれば,$x$は$z$の部分である.
\end{enumerate}


\subparagraph{2.2}
\begin{enumerate}
    \item もし,全く同じ完全性を有する二つの異なる個体が存在するのであれば,それは半順序ではない.
    \item 半順序ではない.たとえば$X = \{x_1, x_2\}, Y = \{y_1, y_2\}$とし,二つの性質関数$f, g: X \to Y$を$f(x_1) = g(x_2) = y_1$かつ$f(x_2) = g(x_1) = y_2$となるように定める.$f$は単射なので,任意の$x, x'\in X$について$f(x) = f(x')$であれば$x=x'$であり,よって$g(x) = g(x')$.この対偶より$g$は$f$に付随する.$g$も同じく単射なので,同様にして$f$は$g$に付随する.しかし$f,g$は$X$の各要素に違う$Y$を割り当てるので$f \neq g$である.よって反対称性を満たさない.
    \item 「$x$が$y$を割り切る」を$x|y$と書くことにする.これは「ある自然数$a$があって$y=ax$」を意味する.$a=1$とすれば,明らかに$x|x$であり反射的.また$x|y$かつ$y|z$ならば, ある自然数$a,b$があって$y = ax$かつ$z = by$であるから,$z = abx$となり$x|z$である,つまり推移的.最後に$y = ax$かつ$x = by$と仮定すると,$y = aby$となり,よって$a=b=1$.したがって$x=y$となり反対称性も満たされる.したがってこの関係は半順序である.
    \item 2.1に前順序としてあげた「$x$は$y$の部分である(ないし$y$に含まれる)」という関係は半順序でもある.というのも,もし$x$が$y$の部分であり,また逆もそうならば,$x$と$y$は同じものだろうからだ.
\end{enumerate}


\subparagraph{3.1}
\begin{enumerate}
    \item $A$: 存在しない.$B: \{x, y\}$.$C$: 存在しない.
    \item $A: \{x,y\}, \{x, y, z\}$. $B: \{x,y\}, \{x, y, z\}$.$C: \{x, y, z\}$.
    \item $A: \{x,y\}$. $B: \{x,y\}$. $C: \{x, y, z\}$.
\end{enumerate}

\subparagraph{3.2}
\begin{enumerate}
    \item 最大元:存在しない.上界:$a$と$b$の公倍数.上限:$a$と$b$の最小公倍数.
    \item 生命の始まりがあるなら,下に有界.いかなる生物もいずれ絶滅すると考えれば,上にも有界.上限は存在しない.下限は$A$の共通祖先.
\end{enumerate}

\subparagraph{4.2}
$x \preceq_X x'$とする.$f$単調より,$f(x) \preceq_Y f(x')$.一方$g$単調より,任意の$y, y'$について$y \preceq_Y y'$ならば$g(y) \preceq_Z g(y')$,よって特に$g\circ f(x) = g(f(x)) \preceq_Z g(f(x')) = g\circ f(x')$.以上より$g \circ f$は単調写像.


\end{document}