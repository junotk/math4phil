\documentclass[11pt,a4paper]{jsarticle}
\usepackage{amsmath,amssymb}
\usepackage{amsthm}
\usepackage{ascmac}
\usepackage{bm}
\usepackage[dvipdfmx]{graphicx}	% required for `\includegraphics' (yatex added)
\usepackage{setspace}           % required for `\doublespace'
\usepackage{tikz}
\usepackage{tikz-cd}
\usetikzlibrary{angles, positioning, shapes, arrows.meta, decorations.pathmorphing}
%\usetikzlibrary{intersections, calc, arrows, positioning, arrows.meta}
\usepackage{tcolorbox}  % 定理環境の装飾
\tcbuselibrary{skins, breakable, theorems}
\usepackage{xcolor}
\usepackage{natbib}
\usepackage{pxrubrica}
\usepackage[margin=30truemm, left=40truemm, right=40truemm]{geometry}
\usepackage{thmbox}     % required for theorem environment with side bar
%
\setlength{\parskip}{3mm} %段落間にスペースを入れる


% \pagestyle{myheadings}
% \markright{\footnotesize \sf 2022秋期「哲学者のための数学」授業資料(大塚淳) \ \ 配布禁止}


\theoremstyle{definition}
\newtheorem[S]{exercise}{練習問題}[section]
\newtheorem[S]{example}{事例}[section]
\newtheorem[S]{fact}{事実}[section]
\newtheorem[S]{attn}{注意}[section]
\newtheorem[S]{develop}{発展}[section]
\renewcommand{\theattn}{}

\newtcbtheorem[auto counter, number within=section]{rei}{事例}{
    breakable,
    coltitle=black,
    fonttitle=\bfseries,
    enhanced, colback=white, frame hidden, borderline west = {0.5pt}{5pt}{black},
%    number freestyle={\noexpand\thesection.\noexpand\arabic{\tcbcounter}}
}{rei}

\newtcbtheorem[auto counter, number within=section]{prop}{命題}{
    breakable,
    coltitle=black,
    fonttitle=\bfseries,
    enhanced, colback=white, frame hidden, borderline west = {0.5pt}{5pt}{black},
%    number freestyle={\noexpand\thesection.\noexpand\arabic{\tcbcounter}}
}{prop}

\newtcbtheorem[number within=section]{renshu}{練習問題}{
    breakable,
    coltitle=black,
    fonttitle=\bfseries,
    enhanced, colback=white, frame hidden, borderline west = {0.5pt}{5pt}{black}
}{renshu}


\newtcbtheorem[number within=section]{hatten}{発展}{
    breakable,
    coltitle=black,
    fonttitle=\bfseries,
    enhanced, colback=white, frame hidden, borderline west = {0.5pt}{5pt}{black}
}{renshu}


\newtcbtheorem[number within=section]{dfn}{定義}{
    fonttitle=\bfseries,
    enhanced, colback=white
}{dfn}


% Bold face capital letters:
\newcommand{\bfzero}{\boldsymbol{0}}
\newcommand{\bfone}{\boldsymbol{1}}
\newcommand{\bfA}{\boldsymbol{A}}
\newcommand{\bfB}{\boldsymbol{B}}
\newcommand{\bfC}{\boldsymbol{C}}
\newcommand{\bfD}{\boldsymbol{D}}
\newcommand{\bfE}{\boldsymbol{E}}
\newcommand{\bfF}{\boldsymbol{F}}
\newcommand{\bfG}{\boldsymbol{G}}
\newcommand{\bfH}{\boldsymbol{H}}
\newcommand{\bfI}{\boldsymbol{I}}
\newcommand{\bfJ}{\boldsymbol{J}}
\newcommand{\bfK}{\boldsymbol{K}}
\newcommand{\bfL}{\boldsymbol{L}}
\newcommand{\bfM}{\boldsymbol{M}}
\newcommand{\bfN}{\boldsymbol{N}}
\newcommand{\bfO}{\boldsymbol{O}}
\newcommand{\bfP}{\boldsymbol{P}}
\newcommand{\bfQ}{\boldsymbol{Q}}
\newcommand{\bfR}{\boldsymbol{R}}
\newcommand{\bfS}{\boldsymbol{S}}
\newcommand{\bfT}{\boldsymbol{T}}
\newcommand{\bfU}{\boldsymbol{U}}
\newcommand{\bfV}{\boldsymbol{V}}
\newcommand{\bfW}{\boldsymbol{W}}
\newcommand{\bfX}{\boldsymbol{X}}
\newcommand{\bfY}{\boldsymbol{Y}}
\newcommand{\bfZ}{\boldsymbol{Z}}

\newcommand{\bfa}{\boldsymbol{a}}
\newcommand{\bfb}{\boldsymbol{b}}
\newcommand{\bfc}{\boldsymbol{c}}
\newcommand{\bfd}{\boldsymbol{d}}
\newcommand{\bfe}{\boldsymbol{e}}
\newcommand{\bff}{\boldsymbol{f}}
\newcommand{\bfk}{\boldsymbol{k}}
\newcommand{\bfm}{\boldsymbol{m}}
\newcommand{\bfn}{\boldsymbol{n}}
\newcommand{\bfo}{\boldsymbol{o}}
\newcommand{\bfp}{\boldsymbol{p}}
\newcommand{\bfq}{\boldsymbol{q}}
\newcommand{\bfr}{\boldsymbol{r}}
\newcommand{\bfs}{\boldsymbol{s}}
\newcommand{\bft}{\boldsymbol{t}}
\newcommand{\bfu}{\boldsymbol{u}}
\newcommand{\bfv}{\boldsymbol{v}}
\newcommand{\bfw}{\boldsymbol{w}}
\newcommand{\bfx}{\boldsymbol{x}}
\newcommand{\bfy}{\boldsymbol{y}}
\newcommand{\bfz}{\boldsymbol{z}}



% BB (???) capital letters:
\newcommand{\bbA}{\mathbb{A}}
\newcommand{\bbB}{\mathbb{B}}
\newcommand{\bbC}{\mathbb{C}}
\newcommand{\bbD}{\mathbb{D}}
\newcommand{\bbE}{\mathbb{E}}
\newcommand{\bbF}{\mathbb{F}}
\newcommand{\bbG}{\mathbb{G}}
\newcommand{\bbI}{\mathbb{I}}
\newcommand{\bbN}{\mathbb{N}}
\newcommand{\bbP}{\mathbb{P}}
\newcommand{\bbQ}{\mathbb{Q}}
\newcommand{\bbR}{\mathbb{R}}
\newcommand{\bbU}{\mathbb{U}}
\newcommand{\bbV}{\mathbb{V}}
\newcommand{\bbX}{\mathbb{X}}
\newcommand{\bbY}{\mathbb{Y}}
\newcommand{\bbZ}{\mathbb{Z}}
\newcommand{\bbone}{{\ifmmode\mathrm{1\!l}\else\mbox{\(\mathrm{1\!l}\)}\fi}}


% Caligraphic math capital letters:
\newcommand{\mcalA}{\mathcal{A}}
\newcommand{\mcalB}{\mathcal{B}}
\newcommand{\mcalC}{\mathcal{C}}
\newcommand{\mcalD}{\mathcal{D}}
\newcommand{\mcalE}{\mathcal{E}}
\newcommand{\mcalF}{\mathcal{F}}
\newcommand{\mcalG}{\mathcal{G}}
\newcommand{\mcalH}{\mathcal{H}}
\newcommand{\mcalI}{\mathcal{I}}
\newcommand{\mcalJ}{\mathcal{J}}
\newcommand{\mcalK}{\mathcal{K}}
\newcommand{\mcalL}{\mathcal{L}}
\newcommand{\mcalM}{\mathcal{M}}
\newcommand{\mcalN}{\mathcal{N}}
\newcommand{\mcalO}{\mathcal{O}}
\newcommand{\mcalP}{\mathcal{P}}
\newcommand{\mcalQ}{\mathcal{Q}}
\newcommand{\mcalS}{\mathcal{S}}
\newcommand{\mcalT}{\mathcal{T}}
\newcommand{\mcalU}{\mathcal{U}}
\newcommand{\mcalV}{\mathcal{V}}
\newcommand{\mcalX}{\mathcal{X}}
\newcommand{\mcalY}{\mathcal{Y}}
\newcommand{\mcalZ}{\mathcal{Z}}

% Graph nodes notations:
\newcommand{\PA}{\mathit{PA}}
\newcommand{\bfPA}{\mathbf{PA}}
\newcommand{\CH}{\mathit{CH}}
\newcommand{\bfCH}{\mathbf{CH}}
\newcommand{\DS}{\mathit{DS}}
\newcommand{\bfDS}{\mathbf{DS}}
\newcommand{\ND}{\mathit{ND}}
\newcommand{\bfND}{\mathbf{ND}}
\newcommand{\AN}{\mathit{an}}
\newcommand{\bfAN}{\mathbf{an}}
\newcommand{\pa}{\mathit{pa}}
\newcommand{\bfpa}{\mathbf{pa}}
\newcommand{\ch}{\mathit{ch}}
\newcommand{\bfch}{\mathbf{ch}}
\newcommand{\ds}{\mathit{ds}}
\newcommand{\bfds}{\mathbf{ds}}
\newcommand{\nd}{\mathit{nd}}
\newcommand{\bfnd}{\mathbf{nd}}
\newcommand{\an}{\mathit{an}}
\newcommand{\bfan}{\mathbf{an}}



\DeclareMathOperator*{\argmax}{arg\,max}
\DeclareMathOperator*{\argmin}{arg\,min}
\DeclareMathOperator*{\argsup}{arg\,sup}
\DeclareMathOperator*{\arginf}{arg\,inf}
\DeclareMathOperator{\erfc}{erfc}
\DeclareMathOperator{\diag}{diag}
\DeclareMathOperator{\cum}{cum}
\DeclareMathOperator{\sgn}{sgn}
\DeclareMathOperator{\tr}{tr}
\DeclareMathOperator{\spn}{span}
\DeclareMathOperator{\adj}{adj}
\DeclareMathOperator{\E}{\mathbb{E}}
\DeclareMathOperator{\var}{Var}
\DeclareMathOperator{\cov}{Cov}
\DeclareMathOperator{\corr}{corr}
\DeclareMathOperator{\sech}{sech}
\DeclareMathOperator{\sinc}{sinc}
\DeclareMathOperator*{\lms}{l.i.m.\,}
\newcommand{\varop}[1]{\var\left[{#1}\right]}
\newcommand{\covop}[2]{\cov\left({#1},{#2}\right)}
\newcommand{\T}{^\textrm{T}}
\newcommand\indep{\protect\mathpalette{\protect\independenT}{\perp}}
\def\independenT#1#2{\mathrel{\rlap{$#1#2$}\mkern2mu{#1#2}}}

\newcommand{\bfalpha}{\boldsymbol{\alpha}}
\newcommand{\bfbeta} {\boldsymbol{\beta}}
\newcommand{\bfgamma}{\boldsymbol{\gamma}}
\newcommand{\bfeta}  {\boldsymbol{\eta}}
\newcommand{\bftheta}{\boldsymbol{\theta}}
\newcommand{\bflambda}   {\boldsymbol{\lambda}}
\newcommand{\bfmu}   {\boldsymbol{\mu}}
\newcommand{\bfnu}   {\boldsymbol{\nu}}
\newcommand{\bfxi}   {\boldsymbol{\xi}}
\newcommand{\bfpsi}  {\boldsymbol{\psi}}
\newcommand{\bfphi}   {\boldsymbol{\phi}}
\newcommand{\bfrho}   {\boldsymbol{\rho}}
\newcommand{\bfvarepsilon}{\boldsymbol{\varepsilon}}
%\newcommand{\qed}{{qed}}
%\newcommand{\eqalignno}[1]{\begin{array}{ccccccc}#1\end{array}}

\newcommand{\bfGamma}{\boldsymbol{\Gamma}}
\newcommand{\bfTheta}{\boldsymbol{\Theta}}
\newcommand{\bfLambda}   {\boldsymbol{\Lambda}}
\newcommand{\bfPsi}  {\boldsymbol{\Psi}}
\newcommand{\bfPhi}   {\boldsymbol{\Phi}}
\newcommand{\bfSigma}  {\boldsymbol{\Sigma}}
\newcommand{\bfOmega}  {\boldsymbol{\Omega}}


% DISTRIBUTIOoNS: 
\newcommand{\normal}{\mathcal{N}}
\newcommand{\binomial}{\mathcal{B}}
\newcommand{\multinomial}{\mathcal{M}}
\newcommand{\exponential}{\mathcal{E}}
\newcommand{\geometric}{\mathcal{G}}
\newcommand{\poisson}{\mbox{Poisson}}
\newcommand{\uniform}{\mbox{Uniform}}

% Logic
\newcommand{\true}{\texttt{true}}
\newcommand{\false}{\texttt{false}}


%PSTricks (commande for latent nodes)
\newcommand{\lnode}[4]{ \cnode(#1){#2}{#3}\rput(#1){\footnotesize#4} }

% KEEPING TRACK OF WORK
\newcommand{\todo}[1]
{
{\color{red}{
[TODO: #1]}}
\addcontentsline{toc}{subsection}{TO DO: #1}
}

\newcommand{\fixme}[1]{{\color{red}{#1}}}

\newenvironment{answer}[1]
{\par \color{blue}{#1}}
{}


\newcommand{\note}[2]
{
{\color{red}{
[#1: #2]}}
}




\makeatletter
% define \citepos for posesive citation (e.g. Otsuka's (2015))
\DeclareRobustCommand\citepos
  {\begingroup
   \let\NAT@nmfmt\NAT@posfmt% ...except with a different name format
   \NAT@swafalse\let\NAT@ctype\z@\NAT@partrue
   \@ifstar{\NAT@fulltrue\NAT@citetp}{\NAT@fullfalse\NAT@citetp}}

\let\NAT@orig@nmfmt\NAT@nmfmt
\def\NAT@posfmt#1{\NAT@orig@nmfmt{#1's}}
\makeatother




% Code for drawing color circle used in topology (pathconnectedness)
\usepackage{xparse}
\ExplSyntaxOn

\keys_define:nn { colour_transition_circle } {
    inner   .fp_set:N   = \l__inner_radius,
    inner   .initial:n  = {2},
    outer   .fp_set:N   = \l__outer_radius,
    outer   .initial:n  = {3},
    angle   .fp_set:N   = \l__start_angle,
    angle   .initial:n  = {0}
}

\NewDocumentCommand \ColourTransitionCircle { O{} m } {
\group_begin:
    \keys_set:nn { colour_transition_circle } {#1}
    \clist_clear:N \l_tmpa_clist
    \clist_map_inline:nn {#2} {
        \clist_put_right:Nn \l_tmpa_clist {##1}
        %\clist_put_right:Nn \l_tmpa_clist {##1}
    }
    \exp_args:Nx \col_trans_circ:n \l_tmpa_clist
\group_end:
}

\cs_new_protected:Npn \col_trans_circ:n #1 {
    \int_step_inline:nnnn {1} {1} {\clist_count:n {#1} - 1} {
        \path[top~color=\clist_item:nn {#1} {##1}, bottom~color=\clist_item:nn {#1} {##1+1}, shading~angle={270-(180-360/\clist_count:n {#1})/2+(##1-1)*360/\clist_count:n {#1}+\fp_use:N \l__start_angle}] ({\fp_use:N \l__inner_radius*cos((##1-1)*360/\clist_count:n {#1}+\fp_use:N \l__start_angle)},{\fp_use:N \l__inner_radius*sin((##1-1)*360/\clist_count:n {#1}+\fp_use:N \l__start_angle)}) arc[radius = \fp_use:N \l__inner_radius, start~angle={(##1-1)*360/\clist_count:n {#1}+\fp_use:N \l__start_angle}, delta~angle=360/\clist_count:n {#1}] -- ({\fp_use:N \l__outer_radius*cos(##1*360/\clist_count:n {#1}+\fp_use:N \l__start_angle)},{\fp_use:N \l__outer_radius*sin(##1*360/\clist_count:n {#1}+\fp_use:N \l__start_angle)}) arc[radius = \fp_use:N \l__outer_radius, start~angle={##1*360/\clist_count:n {#1}+\fp_use:N \l__start_angle}, delta~angle=-360/\clist_count:n {#1}] -- cycle;
    }
    \path[top~color=\clist_item:nn {#1} {\clist_count:n {#1}}, bottom~color=\clist_item:nn {#1} {1}, shading~angle={180-180/\clist_count:n {#1}+\fp_use:N \l__start_angle}]({\fp_use:N \l__inner_radius*cos((\clist_count:n {#1}-1)*360/\clist_count:n {#1}+\fp_use:N \l__start_angle)},{\fp_use:N \l__inner_radius*sin((\clist_count:n {#1}-1)*360/\clist_count:n {#1}+\fp_use:N \l__start_angle)}) arc[radius = \fp_use:N \l__inner_radius, start~angle={(\clist_count:n {#1}-1)*360/\clist_count:n {#1}+\fp_use:N \l__start_angle}, delta~angle=360/\clist_count:n {#1}] -- ({\fp_use:N \l__outer_radius*cos(\clist_count:n {#1}*360/\clist_count:n {#1}+\fp_use:N \l__start_angle)},{\fp_use:N \l__outer_radius*sin(\clist_count:n {#1}*360/\clist_count:n {#1}+\fp_use:N \l__start_angle)}) arc[radius = \fp_use:N \l__outer_radius, start~angle={\clist_count:n {#1}*360/\clist_count:n {#1}+\fp_use:N \l__start_angle}, delta~angle=-360/\clist_count:n {#1}] -- cycle;
}

\ExplSyntaxOff

\begin{document}


\title{0. 準備}
\author{2025秋期「哲学者のための数学」授業資料(大塚淳)}
\date{ver. \today}
\maketitle



\section{定義と具体例}
我々は本授業で,哲学に対する形式的なアプローチを学ぶ.
しかし形式的とはどういうことだろうか.
ここで念頭にあるのは,定義や公理から出発する,数学的な方法である\footnote{数学とは何か,その固有な方法論はどうあるべきか,という大きな問題はここでは全く触れない.}.
一般的な数学の教科書を開いてみると,そこではまずある数学的対象が定義され,そこからその対象が満たす様々な性質が命題ないし定理として演繹的に示されていく.
いきなりだが,一つ例を見てみよう.

\begin{dfn}{前順序}{}
 集合$X$とその上に定義された二項関係$\preceq$が以下の性質を満たす時,組$\langle X, \preceq \rangle$を,前順序(preorder)という.
\begin{enumerate}
 \item $X$のすべての要素$x$について,$x \preceq x$が成り立つ(反射性).
 \item $X$のすべての要素$x, y, z$について,$x \preceq y$かつ$y \preceq z$が成り立つなら,$x \preceq z$も成り立つ(推移性).
\end{enumerate}
\end{dfn}

「集合」「二項関係」などの言葉が出てきたが,とりあえず気にしないでおこう.
とにかくここでは「前順序」なる数学的対象が,性質(1, 2)を満たすような関係$\preceq$を持った集合(ものの集まり)として定義されている.
性質(1, 2)は前順序を構成する関係$\preceq$が満たすべきルールを定めており,「公理」とも呼ばれる.

定義が与えられたら,まずやるべきことはその定義が当てはまる具体例を探すことだ\footnote{本当はその前に,定義が整合的で矛盾がないかどうか,つまりwell-definedであるかどうかをチェックしなければならないが,本講義ではその点は問題にしない.}.
例えば,上で抽象的に定義された「前順序」の具体例としては,どのようなものがあるだろうか?
一つの例としては,自然数上の関係$\leq$が考えられるだろう.
実際,すべての自然数$x$について$x \leq x$だし,任意の3つの自然数$x, y, z$について$x \leq y, y \leq z$なら当然$x \leq z$だ.
よって自然数の集合$\bbN$とその上の関係$\leq$は前順序である.
また同様に,$\bbN$と関係$\geq$も(別の)前順序関係をなす.

しかしこの一つの例だけで前順序をわかった気になるのは危険である.
というのも,例えば自然数では任意の$x,y$について,$x \leq y$あるいは$y \leq x$が成り立つ(これを完備性という,詳しくは3章で扱う)が,こうした規定は前順序には含まれていない.
つまり自然数は一直線に並んでどのペアも大小が比較できるが,前順序は必ずしもそうでなく,「枝分かれ」を許す.
なので,自然数の大小関係という特定の具体例だけで前順序を捉えるべきではない.

また,非常に極端な例もある.
一つの要素「0」のみからなる集合$\{ 0 \}$と,イコール関係$0 = 0$を考えてみると,これは上の公理1, 2を満たす.
よってこれは前順序であるが,我々がイメージする「順序」とは似ても似つかない!
なので定義が与えられたら,一つの具体例で満足するのではなく,様々な例を考えてみること,そしてそれらの例に引きずられないことが重要である.




\section{哲学的問題の形式的モデリング}

哲学的問題を形式的に扱うとは,上でやったような数学的定義を,哲学的問題に当てはめてみる,ということである.
つまり形式的定義の具体例を,哲学の問題から探すということだ.
例えば,中世から近世西洋哲学では,事物の間には「完全性」における優劣関係があると考えられていた.
そこで$X$を事物の集まり,$x \preceq y$を「$y$は$x$と同等かそれ以上に完全である」という関係によって解釈してみると,「完全性という形而上学的関係は,前順序をなす」という一つの仮説ができる.
この仮説がもっともらしいかどうかは,「より完全である」という関係が前順序の公理1, 2を満たすかどうかを考えればよい.つまり
\begin{enumerate}
 \item すべての事物は,それ自身と同等あるいはそれ以上に完全だろうか?
 \item 完全性関係は推移的だろうか?(つまり$y$が$x$以上に完全で,$z$が$y$以上に完全なとき,$z$は$x$以上に完全だろうか?)
\end{enumerate}
これらが肯定的に答えられるのであれば,前順序によって完全性関係をモデル化するのは理に適っている,ということになる.

では,具体的にどうやったら,完全性がこれらの公理を満たすということが示せるのだろうか?
これがモデリングにおける一番のキモであり,これを説得的に示せるかどうかが第一の関門になる.
理想的なのは,哲学者のテキストから,公理が満たされていると判断できるような直接的な証拠を探し出すことである.
例えば誰かが,「あるものは,少なくともそれ自身と同程度には完全である」というようなことを書いていたら,公理1は満たされている(あるいは少なくともその哲学者は,完全性を反射的なものとして理解している)と判断できる.
しかしそれを探すのはしばしば大変だし,運良く見つかるとも限らない.
そこで次善の策は,哲学者が「完全性」概念をどのように用いているかを見て,そこから類推することである.
例えば,もしそれがある種の「量」のように理解されているのであれば,「量」は推移性を満たすので,類比的に完全性も推移性を満たすと判断できる.
この場合,単に完全性を前順序としてモデル化するためには,前者が量のようなものとして扱われている,ということをテキスト上の証拠から示せれば良い,ということになる.
しかしそのような証拠も見いだせない場合がある.
その場合,最後の手段として,「とりあえずこのような公理を満たすものとしてモデル化してみる」という方策があり得る.
つまり,哲学者が本当に完全性を前順序のように考えていたのかは分からないが,とりあえず一つの解釈として,そのように仮定して話を進めてみよう,ということだ.
そんな不確かな仮定の上で話をすすめてどうするんだ,と思うかもしれないが,エビデンスが不足しがちな哲学史研究においては,こうした仮定的な解釈に頼らざるを得ないケースはしばしばある(というよりほとんどそうかもしれない).
その上で,このような仮定的モデリングが,少なくともその哲学者の書いていることと矛盾しない,すなわち整合的なのであれば,一つの仮説としては十分成立しうる.

以上をまとめると,次のようになる.ある数学的道具立てで哲学的概念をモデリングするためには:
\begin{enumerate}
 \item その概念が,当該の数学的対象の公理を満たすということを支持するような,テキスト上の証拠を示す.
 \item それが難しい場合,その概念が,当該の数学的対象の公理を満たすことが分かっているような他の概念と類比的である,ということを示唆するような,テキスト上の証拠を示す.
 \item それも難しい場合,とりいあえずその概念をモデリングした上で,それがその概念についての他の特徴づけと矛盾しないことを示す.
\end{enumerate}

もちろん,哲学モデリングが用いられるのは哲学史的な解釈だけではない.
現行の哲学的問題についても,モデリングが有効なケースはよくある.
哲学は「因果性」や「責任」,あるいは「形而上学的基礎づけ(grounding)」などといった抽象的な概念を議論するが,こうした議論は,その抽象さゆえにしばしば捉えとらえどころがない.
そのようなとき,数学的にモデリングすることで,抽象的な概念に「形をもたせる」ことができる.
例えば,因果関係は前順序だろうか?
一般的な感覚では,モノがそれ自体の原因であるとみなされることはない,つまり反射性は満たされなそうに見えるが,本当にそうだろうか.
また推移性はどうだろうか.
こうした点を検討することで,そもそも「因果関係」とはどのようなものか,ということについての直観をより詳細に詰めていくことができる.

場合によっては,モデリングしていく過程で,ある概念には隠れた前提が含まれていたり,実は矛盾をきたしている,ということが判明するかもしれない.
このようなことは,自然言語で議論している間はなかなか気が付かないことが多い.
数学的に定式化することで初めて,隠された前提や矛盾が明らかになるのである.
よって数学的モデリングは,ある種の\emph{概念実証}(proof of concept)を含んでいる.
概念実証とは,提案された概念が,本当に可能なのかどうか,様々な手段を通して検討することである.
形式モデリングは,哲学的に定義された概念が,そもそも論理的に整合的なのかどうかを,数学という道具立てを用いて「検証」するのである.

さて,以上のような仕方で,何らかの概念が一定の数学的道具立てで表現できた,例えば「完全性」という概念が前順序の公理を満たすと示すことができた,と仮定しよう.
その場合,我々はさらに歩みをすすめて,完全性という関係は前順序によって\emph{十分に}モデル化されているのか,ということを問うことができる.
自然数の大小関係が「単なる」前順序ではなかったように,完全性をモデリングするためにはさらなる公理が必要なのかもしれない.例えば,それは完備なのだろうか?あるいはさらに別の公理が必要なのだろうか?
アンセルムスやデカルトにおいて,完全性は神の存在論的論証で用いられた.
その論証を成立させるためには,どのような仮定が必要なのだろうか.
また,そうした仮定をおいた場合,完全性にはどのような制約が課されるか.
哲学的問題の形式的モデリングでは,こうしたことについて考える必要がある.


\section{形式的モデリングの意義}

以上,本書が考える「哲学モデリング」の基本的な方針を示した.
ここから,そもそもなぜ哲学的概念を数理的にモデリングするのか,その意義みたいなものも少し明らかになったのではないだろうか.
つまり哲学モデリングとは哲学概念を解釈する一つの方法である.
過去の哲学者のテキストや概念を様々な仕方で解釈することは,哲学の重要な仕事の一つである(「イデア」とは,「実体」とは,「超人」とは何か等々).
そうした解釈の主眼は,対象となるテキストを別の仕方で言い換えることにより,元の思考をmake senseすることである.
哲学モデリングのねらいも同様であり,唯一違うのは,「別の仕方での言い換え」に,日本語や英語などの自然言語ではなく,数学という形式言語を用いることでしかない.

ではなぜ,数学を用いて解釈するのだろうか.
一つの利点は,数学的道具立ては,対象の論理的な構造を明らかにするのに向いているということだ.
例えば,完全性を前順序としてモデリングすることで,そこで最低限どのような論理的関係が仮定されているのか(反射性と推移性)ということがわかる.それによって,完全性を用いた議論で何が効いているのか,そして場合によっては,そうした議論にはどのような追加の前提が仮定されているのかを明らかにすることができる.
哲学に見られるような抽象的な議論では,どうしても言葉の「印象」に引きづられて議論が発散したり,ずさんになりがちである.そういうことを防ぐためにも,当該の概念が何を仮定し,何を仮定していないか,ということを明らかにすることは重要である.

第二の利点として,数学的議論を通して,それまで明らかでなかった含意がもたらされたり,議論が明晰になったりする,という可能性がある.
冒頭でも述べたように,数学の強さは,ある一定の公理から出発して,そこから必然的に成り立つ様々な定理を証明していくことにある.
例えば前順序なら前順序,群なら群の定理がある.
よってもし,ある哲学概念がこうした道具立てによってモデリングできる,つまりそれらの公理を満たすことが示せたならば,当然,それに付随する定理もその概念について成り立つことがわかる.
このことにより,それまでは決して自明ではなかった様々な概念的含意が,数学的なルートを通して明らかになる.あるいはいうなれば,数学は自然言語による議論では到底到達できなそうな場所に,我々を連れて行ってくれるのである.

なので,哲学モデリングにも様々な段階がある.
一つは,上にあげたように,ある哲学的概念を数学的概念で解釈する,つまり単に哲学的概念の数学的な表現のみを目的とする段階である.
そして二つ目は,そうしたモデルを数学的に分析することで,これまで知られていなかったような哲学的含意を導き出すことである.
プロの哲学研究(例えば哲学のジャーナルに掲載されるような研究)にとっての形式モデリングの醍醐味は後者の段階であるが,しかしこれはなかなか難しい.
確かに,うまくモデリングすることで哲学上の難問の見通しが良くなったり,あるいは魔法のように氷解したりすることもあるかもしれないが,それは非常に幸運なケースであり,また対象と数学の両方に精通していなければならない.
大抵のモデリング,しかもこの授業でやろうとする初歩的な第一歩では,単に「縦のものを横にする」程度のトリビアルな翻訳くらいしかできない.
しかしだからといって,侮ってはいけない.むしろそうした翻訳には大きな教育的意味がある.
伝統的に,日本における哲学トレーニングの第一歩は翻訳であった.哲学において,翻訳は単に言語的な修練につきるものでは決してない.哲学書の単語を逐一辞書で調べて置き換えても,意味が通る翻訳にはならない.ちゃんとした翻訳をするためには,まず対象となるテキストの意味や文脈を把握し,自分の中で咀嚼・再構成して,翻訳先のテキストを作り上げる必要がある.それは上で述べた解釈の作業にほかならない.
しかしながら,AIが普及した今日,翻訳作業は哲学的解釈のトレーニングとしての役割を終えつつある.
辞書を片手に原典とにらめっこしなくても,AIがそれらしい翻訳を仕上げてくれる(そしてその品質は,少なくとも初学者のそれよりも遥かに良い).
これは教育者にとってというよりも,むしろ哲学を学ぶ者にとって不幸であるように思う.
カントも言うように,判断力は訓練によってのみ身につく,それなのにその訓練の貴重な機会と動機がなくなってしまうのだから!
しかし我々は技術の発展を嘆くのではなく,むしろ新しい訓練の方法を考える必要がある.本書で取り組む哲学モデリングは,そうした新しい訓練の一つの試みである.

もちろん,どのような方法もそうであるように,数学的モデリングには欠点もある.
最も大きな欠点は,議論の論理的な構造のみを抜き出すことで,それ以外の豊かな含意を捨象してしまうことだろう.
多くの哲学の議論において,そうした機微は極めて(おそらく論理的構造以上に)重要である.
なので数学的モデルを,哲学的議論や概念の「本質」と取り違えることは危険である.
実際のところ,多くの場合,そうしたモデルはオリジナルな哲学的思想の貧相な影に過ぎない.
影を本体と取り違えてはいけない.しかし射影することで,初めて明らかになる構造も存在する.
この区別を心に刻んだ上で,できるだけ効果的な仕方で影が生まれるように光を当てる仕方を考える,このことが哲学モデリングを行うにあたって肝要となる.


\section{本書の内容}
上述した目論見のもと,本書では集合・関係・順序・束・位相・モノイド・群という数学的道具立てを扱う.それぞれの箇所で扱われる哲学的問題の例を,以下に示す:
\begin{enumerate}
    \item 集合:意味論,ラッセルパラドクス
    \item 関係と関数:Grounding,因果性,意味論,四次元主義,本質
    \item 順序:Supervenience,不動の第一動者,概念,心身平行論
    \item 束とブール代数:メレオロジー,概念,理論と予測,汎通的規定,反証可能性
    \item 位相:類似性,曖昧性,意味の全体論,検証可能性,不可識別者同一の原理
    \item モノイド:人生,心的感覚,モナド,目的論,決定論と非決定論
    \item 群:タイプとトークン,逆転クオリア,グルーパラドックス,還元主義
\end{enumerate}


\section{命題論理}
本授業では,数学的記号に加え,必要最小限の論理記号を用いる.\footnote{ちなみに本授業では一貫して古典論理を前提とする.}
これらはどれも論理学を少しかじった人にはおなじみだと思うが,ここで確認しておこう.

\begin{itemize}
 \item $\neg P$ 「$P$でない」
 \item $P \vee Q$ 「$P$もしくは$Q$」
 \item $P \wedge Q$ 「$P$かつ$Q$」
 \item $P \Rightarrow Q$ 「$P$ならば$Q$」
 \item $P \iff Q$ 「$P$のとき,その時に限り$Q$」
\end{itemize}

それぞれの真理値は以下の通り

\begin{table}[h]
\centering
\begin{tabular}{cc|ccccc} \hline
 $P$ & $Q$ & $\neg P$ & $P \vee Q$ & $P \wedge Q$ & $P \Rightarrow Q$ & $P \iff Q$ \\ \hline 
 T  & T & F & T & T & T & T \\
 T  & F & F & T & F & F & F \\
 F  & T & T & T & F & T & F \\
 F  & F & T & F & F & T & T \\ \hline
\end{tabular} 
\end{table}

\begin{attn}
「ならば」は,前件が偽になるとき常に真になることに気をつけよう.
\end{attn}

また以下のような基本的ルールは本書に限らずよく使われるので,これを機に確認しておこう.
\begin{itemize}
    \item[(i)] 分配則
    \[ P \wedge ( Q \vee R ) \iff (P \wedge Q) \vee (P \wedge R) \]
    \[ P \vee ( Q \wedge R ) \iff (P \vee Q) \wedge (P \vee R) \]
    \item[(ii)] ド・モルガン則
    \[ \neg (P \vee Q) \iff \neg P \wedge \neg Q \]
    \[ \neg (P \wedge Q) \iff \neg P \vee \neg Q \]
    \item[(iii)] 「ならば」についての同値
    \[ P \Rightarrow Q \iff \neg P \vee Q \iff \neg Q \Rightarrow \neg P \] 
\end{itemize}

これらの論理式は,構成する命題($P, Q, R$)がなんであっても常に成り立つ\emph{トートロジー}である.
実際,上のような真理値表を作れば,左辺の真理値と右辺の真理値の並びが全く等しくなることが確認できる.
両者の真理値が等しいので,「$\iff$」の左辺と右辺はいつでも交換できる.
これを用いて,式変形をしていくというのが常套手段である(例えば「$P$ならば$Q$」を示したいときに,その対偶の「$\neg Q$ならば$\neg P$」を示す,など).


\begin{exercise} 
 \begin{enumerate}
     \item (iii)の各式の真理値を調べ,$\neg P \vee Q$および$\neg Q \Rightarrow \neg P$の真理値が左辺と一致することを確認せよ.
     \item (i)「分配則」の一つ目の左辺と右辺の真理値を調べ,両者が一致することを確認せよ.この場合,$P, Q, R$の三つの要素命題があるので,$2^3=8$行の真理値表が必要になる.
 \end{enumerate}
\end{exercise}

\section{述語記号と量化子}
数学においても哲学においても,基本的な命題は,陰に陽に,何らかのモノについて,それが一定の性質を持つかどうか,という形で述べられる,あるいは一見そうは見えなくとも,そうした命題へと還元できる,ということがしばしばである.
述語論理は,こうした命題を表すための基本的な枠組みを与える.

述語論理は,命題を項と述語に分解する.
例えば「ソクラテスは人間である」という命題は,ソクラテスという個体を表す\emph{定項}$a$と「人間である」という\emph{述語}$H$を用いて$Ha$と表される.
一般的に,個体定項は小文字アルファベット$a, b, \cdots$,述語は大文字$P, Q, \cdots$などで表す.
またさらに,誰それと決め打ちせず,何らかの個体を表す\emph{変項}を小文字の$x, y$などで表すことにする. 
変項が指し示しうる対象の範囲をドメインという(例えば人の集合,物の集合,等々).

述語論理で「述語」とされるものは非常に広い.例えば次のようなものは述語である.
\begin{enumerate}
    \item 「〜は赤い」などの形容詞.e.g. $Ra$.
    \item 「自動車」などの一般名詞.この場合,$Ca$で「$a$は自動車である」などと表す.
    \item 「〜は...を愛する」などの関係一般.e.g. $L(a,b)$ないし$aLb$. 
    \item さらに「$x$と$y$は$z$の両親である」など,3項以上の項をとる述語もある.
    \item 数学的関係は基本的に述語である.例えば「$x \leq y$」という関係は2項述語である.
\end{enumerate}

述語論理を特徴づけるのは,\emph{量化子}の存在である.
これには全称量化と存在量化の二つがある.
\begin{itemize}
 \item 全称量化:$\forall x (Rx)$ 「すべての$x$について述語$R$が成り立つ」
 \item 存在量化:$\exists x (Rx)$ 「ある$x$について述語$R$が成り立つ」
\end{itemize}
たとえば,「すべての人間は死ぬ」という命題は,「人間である」を$H$,「死ぬ」を$D$という述語で表すと,$\forall x (Hx \Rightarrow pDx)$,すなわち「すべての$x$について,それが人間ならば,それは死ぬ」という形で表現できる.

\begin{exercise}
    「すべての哺乳類($M$)が猫($C$)であるわけではない」という文を,全称記号$\forall$を使った述語論理の式として書き下せ.    また同様の文を,存在記号$\exists$を使った式で書き下せ.
\end{exercise}

この練習問題から示唆されるように,全称量化と存在量化の間には次のような関係がある
\[ \neg \forall x Rx \iff \exists x \neg Rx \]
\[ \neg \exists x Rx \iff \forall x \neg Rx \]
実際,「すべての$x$が$R$なのではない」というのは「$R$でない$x$がある」ということと同じ,
また「$R$な$x$はない」というのは「すべての$x$は$R$でない」ということと同じである.

量化子が対象とする範囲については,ドメインで区切る他,$\forall x \in \bbN$のように,$x$が「走る」スコープを明示して制約することもある(ここでの$\bbN$は自然数全体の集合を表す).例えば$\forall x \in \bbN, \exists y \in \bbN (x \leq y)$は,「すべての自然数についてそれより大きい自然数がある」ということを述べている.
ちなみ$\bbN$は自然数,$\bbZ$は整数,$\bbR$は実数の集合を表すこととする.

では練習として,以下の文を述語論理の式で表すことを考えてみよう:「 哲学者というものは,お互いにいがみ合っているものだ.」
まずこの文は,先程の「すべての人間は死ぬ」という例文同様,「哲学者一般」について述べた文であり,しかもそうした哲学者二人についての関係を言っているのだから,$\forall x \forall y...$と二つの全称量化が必要だ.
「哲学者」は一般名なので,一項述語$P$を使おう.
「$x$と$y$が互いにいがみ合う」については,とりあえず二項関係$xRy$で表してみよう.
以上をあわせて,「すべての$x, y$について,両者が哲学者なら,いがみ合う」を表してみると,
\begin{equation}
 \forall x \forall y ((Px \wedge Py) \Rightarrow xRy)
\end{equation}
といったところだろうか.
しかしこのままだと$x$と$y$は同一人物を指しても良いため,自動的に全ての哲学者は自分自身ともいがみ合っていることになってしまう.
「自分自身といがみ合う」ことが可能かはともかく,哲学者は基本自説に甘いのでこれは実情とは異なる.
そこで$x \neq y$という条件を入れて,以下のようになる.
\begin{equation}
 \forall x \forall y (((Px \wedge Py) \wedge (x \neq y)) \Rightarrow xRy)
\end{equation}
これが当面の答えだが,しかしここで立ち止まる必要はない.
例えば「いがみ合う」という関係をより詳細に,「$x$は$y$を批判する$xCy$」という関係を用いて,$xRy \Leftrightarrow xCy \wedge yCx$と定義しても良いかもしれない.
あるいは量化子のスコープ,ないしそれが走るドメインを明確化しても良いかもしれない.

こうしたこと論理学に親しい人であればおなじみの手続きかもしれないが,ここにもすでに「数学的モデリング」の萌芽が見られる.
まずこうした形式言語に書き直すことによって,日常言語では暗黙的な前提(例えば「いがみ合う」という時,通常自分自身とのいがみ合いは除外されていること等)が明らかになる.
次に,こうしたモデリングの仕方や答えは一通りに決まっているわけではなく,複数の異なったアプローチがありうる.
だからといって「なんでもあり」というわけではもちろんなく,良いモデリングと悪いモデリングがある.
我々は今後様々なモデリング手法を学んでいく際,こうしたことに留意しつつ進めることにしよう.


\begin{exercise}
 次の命題の意味を明らかにし,その真偽を確定せよ.
\begin{enumerate}
 \item $\forall x \in \bbZ, \exists y \in \bbZ (y = 0 - x)$
 \item $\exists e \in \bbN, \forall x \in \bbN (x \cdot e = x)$
 \item $\forall x \in \bbN (\sqrt{x} > 3 \Rightarrow x \geq 10)$
\end{enumerate}
\end{exercise}

\begin{exercise}
 定義1.1にあげた前順序の2つの公理を量化子を使って書き下せ.
\end{exercise}





\begin{exercise}
 $xLy$を,「$x$は$y$を愛する」という意味の2項述語としたとき,次を論理式によって書き下せ.
\begin{enumerate}
 \item 誰にも愛する人がいる.
 \item すべての人から愛される人がいる.
 \item すべての人を愛する人なんて存在しない.
 \item 相思相愛のカップルなんて存在しないんですよ.
 \item 好きになった人はことごとく他の人のことが好き,という人がいる.
\end{enumerate}

\end{exercise}


\end{document}